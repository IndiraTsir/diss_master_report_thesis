DRAFT ZERO

\section{Introduction}
The tiger mosquito (\textit{Aedes albopictus}) is an invasive species known for its ability to transmit diseases such as dengue and chikungunya \cite{BONIZZONI2013460}. Since its initial detection in France, this species has rapidly expanded its range due to climate change, creating favorable conditions for its survival. The French population, lacking natural immunity to these diseases, faces significant health risks.

Traditional surveillance methods have proven valuable but are constrained by limited data, resources, and the need for large-scale, real-time monitoring. Recent advances in crowdsensing and citizen science offer a promising solution to these challenges by engaging the public in data collection efforts \cite{Sousa-2022}.

The proliferation of \textit{Aedes albopictus} in Lyon poses serious health and socioeconomic threats. 
Individual health risks arise from increased exposure to mosquito-borne diseases, with cases such as the dengue infection reported in Lyon’s third arrondissement in August 2024 triggering immediate vector control responses \cite{dengue_lyon}. Furthermore, the aggressive daytime biting behavior of this mosquito reduces outdoor activities, diminishing personal well-being and comfort.

At the community level, the economic and social impact is substantial. The disruption of outdoor commerce, tourism, and public spaces affects local businesses and urban lifestyles \cite{haderer:hal-00783873}. Schools, parks, and neighborhoods experience reduced livability, necessitating community-driven responses to mosquito control.
On a broader scale, the increasing presence of 	extit{Aedes albopictus} in France poses nationwide public health risks, necessitating the development of scalable intervention models that could serve as templates for other regions facing similar threats \cite{article}.





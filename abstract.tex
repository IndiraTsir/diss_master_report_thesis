AVEC 2 CITATIONS CLIMAT et difference sur methodes application


\begin{abstract}
The rapid expansion of \textit{Aedes albopictus} (tiger mosquito) in Lyon, France, presents a significant public health challenge exacerbated by climate change\cite{paris, IPCC_2022_WGIII_SPM}. This invasive species can transmit diseases such as dengue and chikungunya, posing a threat to the immunologically naive French population\cite{10_24072_pcjournal_326}. Despite past efforts using traditional surveillance methods, limited resources and data availability have hindered effective control strategies. Addressing this challenge with innovative interdisciplinary\cite{isprs-archives-XLVIII-4-2024-397-2024} methods could inform more effective mosquito control strategies, benefiting Lyon and other regions around the world.
We maneuvered this initiative through a methodological collection of sewer water samples of the French city - Villeurbanne, (ref Axelle et Claire comment?) and investigated its physicochemical properties combined with the micro-organisms and micro-polluants variables that were detected inside. The objectives of this research included several points: (1) understand how the microbial composition of larval habitats covaried differentially along the pollution gradients according to colonization status, (2) identify the influence of biotic and abiotic interactions, (3) understand the colonization of mosquito larval habitats according to their biotic and abiotic characteristics with graph neural networks. Those helping interdisciplinary research and improve public health preparedness globally.
\end{abstract}
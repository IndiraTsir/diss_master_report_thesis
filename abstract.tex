\begin{abstract}
The rapid expansion of \textit{Aedes albopictus} (tiger mosquito) in Lyon, France, presents a significant public health challenge exacerbated by climate change\cite{10.1093/femsec/fiae129}. This invasive species can transmit diseases such as dengue and chikungunya, posing a threat to the immunologically naive French population\cite{10_24072_pcjournal_326}. Despite past efforts using traditional surveillance methods, limited resources and data availability have hindered effective control strategies. Addressing this challenge with innovative \textbf{interdisciplinary} methods could inform more effective mosquito control strategies, benefiting Lyon and other regions around the world.
We propose an initiative that uses a water sample collection done by a PhD student that have worked on this theme for two years (Axelle Gentil) and a supplementary analysis performed by a Master's student in informatics. The objectives of this research include: (1) gain more understanding of the dataset through the techniques of data analysis and data science, (2) identify what can attract mosquitoes and create a breeding site, (3) utilizing AI and data science techniques to analyze mosquito dynamics, and (4) understand the colonization of mosquito larval habitats according to environmental factors.
Key challenges such as data size, quality and validation of scientific methods are addressed. Using advanced data science techniques, the collected data will inform strategic research to improve public health preparedness. This research has the potential to create an interdisciplinary step by step guide for global researchers, contributing significantly to combating mosquito-borne diseases and helping merge different disciplines.
\end{abstract}

MODIF A FAIRE: INCLURE DES ARTICLES AVEC LA PLURIDISCIPLINARITE (peut etre celle de remy claire et penelope est ok?)
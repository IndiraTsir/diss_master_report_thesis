\documentclass[acmlarge]{acmart}

\AtBeginDocument{%
  \providecommand\BibTeX{{%
    Bib\TeX}}}

\setcopyright{}
\acmYear{2025}
\acmDOI{}

\begin{document}

\title{A Crowdsensing Approach to Monitor and Control the Spread of Aedes albopictus in Lyon, France}

\author{Indira Tsirikhova}
\affiliation{%
  \institution{Université Claude Bernard Lyon 1}
  \city{Lyon}
  \state{Auvergne-Rhône-Alpes}
  \country{France}
}

\author{Remy Cazabet}
\affiliation{%
  \institution{UCBL}
  \city{Lyon}
  \state{Auvergne-Rhône-Alpes}
  \country{France}
}

\author{Claire Valiente Moro}
\affiliation{%
  \institution{UCBL}
  \city{Lyon}
  \state{Auvergne-Rhône-Alpes}
  \country{France}
}

\author{Axelle Gentil}
\affiliation{%
  \institution{UCBL}
  \city{Lyon}
  \state{Auvergne-Rhône-Alpes}
  \country{France}
}

\begin{abstract}
The rapid expansion of \textit{Aedes albopictus} (tiger mosquito) in Lyon, France, presents a significant public health challenge exacerbated by climate change\cite{10.1093/femsec/fiae129}. This invasive species can transmit diseases such as dengue and chikungunya, posing a threat to the immunologically naive French population\cite{10_24072_pcjournal_326}. Despite past efforts using traditional surveillance methods, limited resources and data availability have hindered effective control strategies. Addressing this challenge with innovative \textbf{interdisciplinary} methods could inform more effective mosquito control strategies, benefiting Lyon and other regions around the world.
We propose an initiative that uses a water sample collection done by a PhD student that have worked on this theme for two years (Axelle Gentil) and a supplementary analysis performed by a Master's student in informatics. The objectives of this research include: (1) gain more understanding of the dataset through the techniques of data analysis and data science, (2) identify what can attract mosquitoes and create a breeding site, (3) utilizing AI and data science techniques to analyze mosquito dynamics, and (4) understand the colonization of mosquito larval habitats according to environmental factors.
Key challenges such as data size, quality and validation of scientific methods are addressed. Using advanced data science techniques, the collected data will inform strategic research to improve public health preparedness. This research has the potential to create an interdisciplinary step by step guide for global researchers, contributing significantly to combating mosquito-borne diseases and helping merge different disciplines\cite{ijerph19106337}\cite{zivko_juznic_zonta_2022_6534797}.
\end{abstract}

\maketitle

\section{Introduction}
The tiger mosquito (\textit{Aedes albopictus}) is an invasive species known for its ability to transmit diseases such as dengue and chikungunya \cite{BONIZZONI2013460}. Since its initial detection in France, this species has rapidly expanded its range due to climate change, creating favorable conditions for its survival. The French population, lacking natural immunity to these diseases, faces significant health risks.

Traditional surveillance methods have proven valuable but are constrained by limited data, resources, and the need for large-scale, real-time monitoring. Recent advances in crowdsensing and citizen science offer a promising solution to these challenges by engaging the public in data collection efforts \cite{Sousa-2022}.

The proliferation of \textit{Aedes albopictus} in Lyon poses serious health and socioeconomic threats. 
Individual health risks arise from increased exposure to mosquito-borne diseases, with cases such as the dengue infection reported in Lyon’s third arrondissement in August 2024 triggering immediate vector control responses \cite{dengue_lyon}. Furthermore, the aggressive daytime biting behavior of this mosquito reduces outdoor activities, diminishing personal well-being and comfort.

At the community level, the economic and social impact is substantial. The disruption of outdoor commerce, tourism, and public spaces affects local businesses and urban lifestyles \cite{haderer:hal-00783873}. Schools, parks, and neighborhoods experience reduced livability, necessitating community-driven responses to mosquito control.
On a broader scale, the increasing presence of 	extit{Aedes albopictus} in France poses nationwide public health risks, necessitating the development of scalable intervention models that could serve as templates for other regions facing similar threats \cite{article}.


\section{Ideas to check}

\begin{itemize}
    \item 1) Intro - objective - challenges - project importance 
    \item 2) Background and related work / State of the art about variables affecting mosquitoes - penelope study and something else??? How they tried to find those things... Our approach is NEW AND WAS NEVER DONE (allegedly and to the best of myyyy knowledge) But talk about what are OTUs, how treated what "read" means VS abiotic. How they have familys and genus and etc etc TOOO MUCH INFO but at the same time not enough. A unique problem (me know nothing, will try to find things with carefull orientation from them, without disclosing the results they have that might have influenced their workflow or their hypotheses. UNIQUEEEE! MUCH WOW!! BE IMPRESSED!!) They do RRRRRR! They have all the code to basically cluster it in a biological manner by family etc etc. But I do it with python, no family influence that will be perceived by my algorithms, every feature will be treated as completely different. Is it good or bad? Don't know, since no previous data on that (unique study remeber!). So how to bundle biotic things with abiotic things and just analyze them all? Well that's our methodology. 
    \item 3) Methodology - EDA, then ML, then go simple analysis with descriptive stats: similar, Dissimilar? Processing, the NA, the temps the things to filter and why. Then test expe. Then test Expe Penelope?? 
    \item 4) Evaluation and results : will see
    \item 5) Conclusion
    \item 
\end{itemize}


\section{Plan Objectives INTRO}
\begin{enumerate}
    \item Tiger mosquito - what is it, why dangerous (vector and diseases) and the impact on general health and the life of people (no can go outside without some spray etc etc)
    \item Why we do this research - ANR SERIOUS with Claire and Axelle - interdisciplinary project so Remy have joined
    \item Me, charged with this project as a way to do what exactly? (data analytics, data science, networks etc etc etc)
    \item The data that we have : collected samples from sewers: 6 of them and 10 times
    \item Experiments that have been also conducted by Axelle on some samples (gites)
    \item What we try to find here : why the hell mosquitoes love some places and not the others and what does that mean for us
    \item We try to understand data, separate abiotics, bacteries, fungis and analyse it
    \item See how similar or dissimilar they are
    \item See if ML or AI can find something
    \item Compare with dumb simple methods like desriptive statistics etc
    \item Gain more familiarity with the dataset, understand better with the "experts" what that may or may not mean (frustrations and challenges here)
    \item What have been found so far
\end{enumerate}

\section{HYPOTHESES}
\begin{enumerate}
    \item similar
    \item not similar
    \item test
    \item networks??
    \item give tools to understand it better with apps like PBI??? (sad)
    \item Compare with DATA PENELOPE (SHOULD DO IT OR IF NO FINDINGS ME DONE!!!!!)
\end{enumerate}



\section{Challenges and Mitigation}
\textbf{Data Size:} tiny.  
\textbf{Data Quality:} errr.  
\textbf{Data Understanding:} me problems.


\section{Impact}
This research addresses the urgent need for innovative mosquito control strategies.


\subsection{Advancements in the Proposed Lyon Study}
This research addresses these gaps through:
\begin{itemize}
\item \textbf{Enhanced Participation Framework:}
\begin{itemize}
\item Targeted recruitment: Biology students at Université Claude Bernard Lyon 1 ensure baseline taxonomic training, reducing misidentification risks.
\item Gamified retention: Achievement badges and academic credits incentivize sustained engagement, addressing attrition observed in Spain.
\end{itemize}
\item \textbf{Data Quality Innovations:}
\begin{itemize}
\item Protocol standardization: App-guided water sampling minimizes collection errors.
\item Triangulated validation: Crowdsourced data is cross-checked by experts.
\end{itemize}
\item \textbf{Methodological Improvements:}
\begin{itemize}
\item Climate-resilient modeling: AI integrates real-time temperature/humidity projections to predict breeding site suitability under climate change.
\item Privacy-preserving granularity: GPS metadata is anonymized without sacrificing spatial resolution, enabling precinct-level interventions.
\item Open-access dataset: Longitudinal mosquito distribution records will facilitate global research on disease-mosquito-climate linkages.
\end{itemize}
\end{itemize}












\bibliographystyle{ACM-Reference-Format}
\bibliography{references}

\end{document}



